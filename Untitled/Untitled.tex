% Options for packages loaded elsewhere
\PassOptionsToPackage{unicode}{hyperref}
\PassOptionsToPackage{hyphens}{url}
\PassOptionsToPackage{dvipsnames,svgnames*,x11names*}{xcolor}
%
\documentclass[
  12pt,
]{article}
\usepackage{lmodern}
\usepackage{amsmath}
\usepackage{ifxetex,ifluatex}
\ifnum 0\ifxetex 1\fi\ifluatex 1\fi=0 % if pdftex
  \usepackage[T1]{fontenc}
  \usepackage[utf8]{inputenc}
  \usepackage{textcomp} % provide euro and other symbols
  \usepackage{amssymb}
\else % if luatex or xetex
  \usepackage{unicode-math}
  \defaultfontfeatures{Scale=MatchLowercase}
  \defaultfontfeatures[\rmfamily]{Ligatures=TeX,Scale=1}
\fi
% Use upquote if available, for straight quotes in verbatim environments
\IfFileExists{upquote.sty}{\usepackage{upquote}}{}
\IfFileExists{microtype.sty}{% use microtype if available
  \usepackage[]{microtype}
  \UseMicrotypeSet[protrusion]{basicmath} % disable protrusion for tt fonts
}{}
\usepackage{xcolor}
\IfFileExists{xurl.sty}{\usepackage{xurl}}{} % add URL line breaks if available
\IfFileExists{bookmark.sty}{\usepackage{bookmark}}{\usepackage{hyperref}}
\hypersetup{
  pdfauthor={author one; author two},
  colorlinks=true,
  linkcolor=blue,
  filecolor=Maroon,
  citecolor=green,
  urlcolor=red,
  pdfcreator={LaTeX via pandoc}}
\urlstyle{same} % disable monospaced font for URLs
\usepackage[margin=1in]{geometry}
\usepackage{longtable,booktabs}
\usepackage{calc} % for calculating minipage widths
% Correct order of tables after \paragraph or \subparagraph
\usepackage{etoolbox}
\makeatletter
\patchcmd\longtable{\par}{\if@noskipsec\mbox{}\fi\par}{}{}
\makeatother
% Allow footnotes in longtable head/foot
\IfFileExists{footnotehyper.sty}{\usepackage{footnotehyper}}{\usepackage{footnote}}
\makesavenoteenv{longtable}
\usepackage{graphicx}
\makeatletter
\def\maxwidth{\ifdim\Gin@nat@width>\linewidth\linewidth\else\Gin@nat@width\fi}
\def\maxheight{\ifdim\Gin@nat@height>\textheight\textheight\else\Gin@nat@height\fi}
\makeatother
% Scale images if necessary, so that they will not overflow the page
% margins by default, and it is still possible to overwrite the defaults
% using explicit options in \includegraphics[width, height, ...]{}
\setkeys{Gin}{width=\maxwidth,height=\maxheight,keepaspectratio}
% Set default figure placement to htbp
\makeatletter
\def\fps@figure{htbp}
\makeatother
\setlength{\emergencystretch}{3em} % prevent overfull lines
\providecommand{\tightlist}{%
  \setlength{\itemsep}{0pt}\setlength{\parskip}{0pt}}
\setcounter{secnumdepth}{5}
% Suppress indenting after some environments
\IfFileExists{noindentafter.sty}{%
  \usepackage{noindentafter}
  \NoIndentAfterEnv{itemize}
  \NoIndentAfterEnv{enumerate}
  \NoIndentAfterEnv{description}
  \NoIndentAfterEnv{Shaded}
  \NoIndentAfterEnv{center}
  \NoIndentAfterEnv{verbatim}
}{}
\IfFileExists{subfig.sty}{%
  \usepackage{subfig}%
}{}
\usepackage{setspace}\singlespacing
\usepackage{indentfirst}
\parskip=3pt
\frenchspacing
\usepackage[sf,bf]{titlesec}
\pagenumbering{gobble}
\usepackage[super]{nth}
\renewenvironment{figure}[1][2] {
    \expandafter\origfigure\expandafter[H]
} {
    \endorigfigure
}
%\usepackage{endnotes} # if journal requires endnotes
%\let\footnote=\endnote
%\usepackage{lineno} #if journal requires line numbering, uncomment these two lines
%\linenumbers
\setsansfont [ Path = fonts/,
  UprightFont = Montserrat-Regular.ttf,
  BoldFont = Montserrat-Bold.ttf,
  ItalicFont = Montserrat-Italic.ttf,
  BoldItalicFont = Montserrat-BoldItalic.ttf ] {Montserrat}

\setmainfont [ Path = fonts/,
  UprightFont = AlegreyaSans-Regular.ttf,
  BoldFont = AlegreyaSans-Bold.ttf,
  ItalicFont = AlegreyaSans-Italic.ttf,
  BoldItalicFont = AlegreyaSans-BoldItalic.ttf ] {EBGaramond}

\setmonofont [ Path = fonts/,
  UprightFont = FiraMono-Regular.ttf,
  BoldFont = FiraMono-Bold.ttf ] {FiraMono}

\makeatletter
\renewenvironment{table}%
  {\renewcommand\familydefault\sfdefault
   \@float{table}}
  {\end@float}
\makeatother
\usepackage{booktabs}
\usepackage{longtable}
\usepackage{array}
\usepackage{multirow}
\usepackage{wrapfig}
\usepackage{float}
\usepackage{colortbl}
\usepackage{pdflscape}
\usepackage{tabu}
\usepackage{threeparttable}
\usepackage{threeparttablex}
\usepackage[normalem]{ulem}
\usepackage{makecell}
\usepackage{xcolor}
\ifluatex
  \usepackage{selnolig}  % disable illegal ligatures
\fi

\title{\sffamily{``Title''}}
% Author block
\IfFileExists{authblk.sty}{%
  \usepackage[]{authblk}%
  \renewcommand\Affilfont{\itshape\small}
  \renewcommand\Authfont{}
  \author{author one}
  \author{author two}
  \affil{affiliation one}
  \affil{affiliation two}
}{
  \author{author one}
  \author{author two}
}
\date{2021-05-23}

\begin{document}
\maketitle

\newpage

\hypertarget{Abstract}{%
\section*{Abstract}\label{Abstract}}
\addcontentsline{toc}{section}{Abstract}

\noindent The abstract should specify the key question that the article addresses. It should also offer a theoretical and conceptual framework for understanding the issue, and for generalizing to other, similar questions. Finally, it should note the methodological or analytical approach to addressing the question, and give a very brief summation of the key finding.

\noindent \textbf{Keywords:} 3--6 key words that allow researchers to find the article

\newpage

\pagenumbering{arabic}

\hypertarget{introduction}{%
\section*{Introduction}\label{introduction}}
\addcontentsline{toc}{section}{Introduction}

The introduction should reintroduce the key question being addressed in the paper. Introductions should also address the importance for furthering the theory or conceptual frame for understanding this problem, and `like' problems. This should also be extend to the problems importance in real-world understanding of human, social, or institutional behavior. Finally, the introduction offers a hook to keep the reader progressing through the article. Add all addition sections as needed.

My workflow depends on the r packages \texttt{tidyverse}, \texttt{kableExtra}, \texttt{readxl}, and \texttt{citr}. Change these dependencies as it fits your needs.

The html below makes sure that references begin on a separate, new page. It also adjusts the indenting and spacing for proper formatting and alignment. The final bit places the references. Simply uncomment to make active (remove \texttt{\textless{}!-\/-\ -\/-\textgreater{}}). I have added placeholder sections that are fairly standard for social science research. Adjust these as necessary.

\hypertarget{background-and-literature}{%
\section*{Background and Literature}\label{background-and-literature}}
\addcontentsline{toc}{section}{Background and Literature}

\hypertarget{theory-conceptualization}{%
\section*{Theory \& Conceptualization}\label{theory-conceptualization}}
\addcontentsline{toc}{section}{Theory \& Conceptualization}

\hypertarget{research-design}{%
\section*{Research Design}\label{research-design}}
\addcontentsline{toc}{section}{Research Design}

\hypertarget{data}{%
\subsection*{Data}\label{data}}
\addcontentsline{toc}{subsection}{Data}

\hypertarget{operationalization-measurement}{%
\subsubsection*{Operationalization \& Measurement}\label{operationalization-measurement}}
\addcontentsline{toc}{subsubsection}{Operationalization \& Measurement}

\hypertarget{methodological-approach}{%
\subsection*{Methodological Approach}\label{methodological-approach}}
\addcontentsline{toc}{subsection}{Methodological Approach}

\hypertarget{analysis-findings}{%
\section*{Analysis \& Findings}\label{analysis-findings}}
\addcontentsline{toc}{section}{Analysis \& Findings}

\hypertarget{conclusion}{%
\section*{Conclusion}\label{conclusion}}
\addcontentsline{toc}{section}{Conclusion}

\end{document}
